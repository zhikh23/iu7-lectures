\section{Уравнение плоскости}

\subsection{Способы задания плоскости}

\subsubsection{Через три точки}

Пусть заданы точки $M_1(x_1, y_1, z_1), M_2(x_2, y_2, z_2), M_3(x_3, y_3, z_3)$, которые принадлежат плоскости $\alpha$. % Через мат символы

\begin{gather*}
  M_1(x_1, y_1, z_1) \in \alpha \\
  M_2(x_2, y_2, z_2) \in \alpha \\
  M_3(x_3, y_3, z_3) \in \alpha \\
\end{gather*}

Выберем точку на плоскости $\alpha$ точку $M(x, y, z)$.

Составим вектора:
\begin{gather*}
  \overrightarrow{M_1 M} = \{x - x_1, y - y_1, z - z_1\} \\
  \overrightarrow{M_1 M_2} = \{x_2 - x_1, y_2 - y_1, z_2 - z_1\} \\ 
  \overrightarrow{M_1 M_3} = \{x_1 - x_3, y - y_3, z - z_3\} \\
\end{gather*}

$\overrightarrow{M_1 M}, \overrightarrow{M_1 M_2}, \overrightarrow{M_1 M_3}$ - компанарны, а значит: \[
\overrightarrow{M_1 M} \cdot \overrightarrow{M_1 M_2} \cdot \overrightarrow{M_1 M_3} = 0
\] 
Следовательно:
\begin{gather*}
  \boxed{
  \begin{vmatrix}
    x - x_1 & y - y_1 & z - z_1 \\
    x_2 - x_1 & y_2 - y_1 & z_2 - z_1 \\
    x_3 - x_1 & y_3 - y_1 & z_3 - z_1 \\
  \end{vmatrix} = 0}
\end{gather*}

\subsubsection{Через две точки с направляющим вектором}

Пусть даны:
\begin{gather*}
  M_1(x_1, y_1, z_1) \in \alpha \\
  M_2(x_2, y_2, z_2) \in \alpha \\
  \vec{S} = \{m, n, p\} \in \beta \\
  \alpha \parallel \beta
\end{gather*}

Выберем на плоскости $\alpha$ произвольную точку $M$  \[
M(x, y, z) \in \alpha
\] 

Составим вектора $\overrightarrow{M_1 M}, \overrightarrow{M_1 M_2}$:
\begin{gather*}
  \overrightarrow{M_1 M} = \{x - x_1, y - y_1, z - z_1\} \\
  \overrightarrow{M_1 M_2} = \{x_2 - x_1, y_2 - y_1, z_2 - z_1\} 
\end{gather*}

Тогда вектора $\overrightarrow{M_1, M}, \overrightarrow{M_1, M_2}, \vec{S}$ - компланарны, а следовательно:
\begin{gather*}
  \overrightarrow{M_1 M} \cdot \overrightarrow{M_1 M_2} \cdot \vec{S} = 0
  \implies \boxed{
  \begin{vmatrix}
    x - x_1 & y - y_1 & z - z_1 \\
    x_2 - x_1 & y_2 - y_1 & z_2 - z_1 \\
       m    &     n    &    p
   \end{vmatrix} = 0}
\end{gather*}

\subsubsection{Проходящей через точку с двумя направляющими векторами}

Пусть даны:
\begin{gather*}
  M_1(x_1, y_1, z_1) \in \alpha \\
  \vec{S_1} = \{m_1, n_1, p_1\} \in \beta \\
  \vec{S_2} = \{m_2, n_2, p_2\} \in \beta \\
  \alpha \parallel \beta
\end{gather*}

Выберем на плоскости $\alpha$ произвольную точку $M$  \[
M(x, y, z) \in \alpha
\] 

Составим вектора $\overrightarrow{M_1 M}$:
\begin{gather*}
  \overrightarrow{M_1 M} = \{x - x_1, y - y_1, z - z_1\} 
\end{gather*}

Тогда вектора $\overrightarrow{M_1, M}, \vec{S_1}, \vec{S_2}$ - компланарны, а следовательно:
\begin{gather*}
  \overrightarrow{M_1 M} \cdot \vec{S_1} \cdot \vec{S_2} = 0
  \implies \boxed{
  \begin{vmatrix}
    x - x_1 & y - y_1 & z - z_1 \\
    m_1 & n_1 & p_1 \\
    m_2 & n_2 & p_2
  \end{vmatrix} = 0}
\end{gather*}

\subsubsection{Уравнение плоскости в отрезках}

Пусть плоскость $\alpha$ отсекает от координатного угла отрезки $a, b, c$ на осях  $x, y, z$ соответственно. Обозначим точки пересечения  $A, B, C$. Тогда: \[
  A(a, 0, 0) \quad B(0, b, 0) \quad C(0, 0, c)
\] 

Выберем на плоскости $\alpha$ произвольную точку $M$  \[
M(x, y, z) \in \alpha
\] 

Составим вектора:
\begin{gather*}
  \overrightarrow{AM} = \{x - a, y, z\} \\
  \overrightarrow{AB} = \{-a, b, 0\} \\
  \overrightarrow{AC} = \{-a , 0, c\} 
\end{gather*}

Тогда вектора $\overrightarrow{AM}, \overrightarrow{AB}, \overrightarrow{AC}$ - компланарны, а следовательно:
\begin{gather*}
  \overrightarrow{AM} \cdot \overrightarrow{AB} \cdot \overrightarrow{Ac} = 0
  \implies
  \begin{vmatrix}
    x - a & y & z \\
    -a & b & 0 \\
    -a & 0 & c
  \end{vmatrix} = 0 \implies \\
  (x - a) \cdot (-1)^{1+1} 
  \begin{vmatrix}
    b & 0 \\
    0 & c
  \end{vmatrix}
  + y \cdot (-1)^{1+2}
  \begin{vmatrix}
    -a & 0 \\
    -a & c
  \end{vmatrix}
  + z \cdot (-1)^{1+3}
  \begin{vmatrix}
    -a & b \\
    -a & 0
  \end{vmatrix} = 0 \implies \\
  (x - a) bc - y(-ac) + zab = 0 \\
  xbc + yac + zab = abc \\
  \boxed{\frac{x}{a} + \frac{y}{b} + \frac{z}{c} = 1}
\end{gather*}

\subsubsection{Общее уравнение}

Пусть даны:
\begin{gather*}
  M_0(x_0, y_0, z_0) \in \alpha \\
  \vec{n} = \{A,B, C\} \text{ - вектор нормали} 
\end{gather*}

Выберем на плоскости $\alpha$ произвольную точку $M$  \[
M(x, y, z) \in \alpha
\] 

Составим вектор $\overrightarrow{M_0 M}$: \[
\overrightarrow{M_0 M} = \{x - x_0, y - y_0, z - z_0\} 
\] 

Тогда:
\begin{gather*}
  \vec{n} \perp \overrightarrow{M_0 M} \implies \vec{n} \cdot \overrightarrow{M_0M} \\
  \iff A(x - x_0) + B(y - y_0) + C(z - z_0) = 0 \\
  \iff Ax + By + Cz + (-Ax_0 - By_0 - Cz_0) = D \\
  \boxed{Ax + By + Cz + D = 0}
\end{gather*}

\subsection{Угол между плоскостями}

Пусть заданы плоскости общими уравнениями:
\begin{gather*}
  \alpha_1: A_1x + B_1y + Cz_1 + D_1 = 0 \implies \vec{n_1} = \{A_1, B_1, C_1\}  \\
  \alpha_2: A_2x + B_2y + Cz_2 + D_2 = 0 \implies \vec{n_2} = \{A_2, B_2, C_2\} 
\end{gather*}

Угол между плоскостями $\alpha_1, \alpha_2$ равен углу между нормалями $n_1, n_2$ к этим плоскостям.

Тогда можно найти:
\begin{gather*}
  \cos \varphi = \frac{\vec{n_1} \cdot \vec{n_2}}{|\vec{n_1}| \cdot |\vec{n_2}|} = \\
  = \boxed{\frac{A_1 A_2 + B_1 B_2 + C_1 C_2}{\sqrt{A_1^2 + B_1^2 + C_1^2} + \sqrt{A_2^2 + B_2^2 + C_2^2}}}
\end{gather*}

\subsubsection{Условие перпендикулярности}

Если $\alpha_1 \perp \alpha_2$, то $\vec{n_1} \perp \vec{n_2} \implies \vec{n_1} \cdot \vec{n_2} = 0 \implies$ \[
  \boxed{A_1 A_2 + B_1 B_2 + C_1 C_2 = 0}
\]

\subsubsection{Условие параллельности}

Если $\alpha_1 \parallel \alpha_2$, то $\vec{n_1} \parallel \vec{n_2} \implies$ \[
  \boxed{\frac{A_1}{A_2} = \frac{B_1}{B_2} = \frac{C_1}{C_2}}
\]

\begin{note}
  Если $\frac{A_1}{A_2} = \frac{B_1}{B_2} = \frac{C_1}{C_2} = \frac{D_1}{D_2}$, то плоскости \textbf{совпадают}.
\end{note}
\begin{note}
  Если $\frac{A_1}{A_2} = \frac{B_1}{B_2} = \frac{C_1}{C_2} \neq \frac{D_1}{D_2}$, то плоскости \textbf{не совпадают}.
\end{note}

\subsection{Расстояние от точки до прямой}

