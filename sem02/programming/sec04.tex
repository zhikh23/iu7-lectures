\section{Одномерные статические массивы}

\begin{definition}
  \textit{Массивом} называется последовательность элементов одного типа, расположенных в памяти последовательно.
\end{definition}

\subsection{Массивы в Си}

Статический массив объявляется как:
\begin{minted}{c}
int arr[10];
\end{minted}

\begin{itemize}
  \item Тип массива может быть любым.
  \item Количество элементов является целочисленным константным выражением.
  \item Размер массивы не может быть изменён в ходе выполнения программы.
\end{itemize}

\subsubsection{Доступ к элементу массива}

Доступ к элементу массива осуществляется операцией \textit{индексации}.

\begin{minted}{c}
int arr[10];
...
int a = arr[2]; // Третий элемент массива
\end{minted}

\begin{itemize}
  \item Для доступа к элементу массива используется индекс.
  \item Индексация выполняется с нуля.
  \item В качестве индекса может выступать целочисленное выражение.
  \item Си не предусматривает никаких проверок на выход за пределы массива.
\end{itemize}

\subsubsection{Инициализация массива}

\begin{minted}{c}
arr[] = { 1, 2, 3 };    // { 1, 2, 3 }
arr[5] = { 1, 2, 3 };   // { 1, 2, 3, 4, 5 }
arr[5] = { 0 };         // { 0, 0, 0, 0, 0 }
arr[5] = { [1] = 1, [3] = 3 }; // {0, 1, 0, 3, 0 }
\end{minted}

