\section{Линейные подпространства}

\begin{definition}[Линейное подпространства]\label{def:21}
  Подмножество $\mathcal{H}$ линейного пространства $\mathcal{L}$ называют \textit{линейным подпространством}, если выполнены следующие два условия:
  \begin{enumerate}
    \item Сумма любых двух векторов из $\mathcal{H}$ принадлежит $\mathcal{H}: x, y \in \mathcal{H} \implies x + y \in \mathcal{H}$;
    \item Произведение любого вектора из $\mathcal{H}$ на любое действительное число снова принадлежит $\mathcal{H}: x \in \mathcal{H}, \lambda \in \R \implies \lambda x \in \mathcal{H}$.
  \end{enumerate}
\end{definition}

Определение \ref{def:21} фактически говорит о том, что линейное подпространство -- это любое подмножество данного линейного пространства, замкнутое относительно линейных операций, т.е. применение линейных операций к векторам, принадлежащим этому подмножеству, не выводит результат за пределы подмножества.

В любом линейном пространстве $\mathcal{L}$ всегда имеются два линейных подпространства: само линейное пространство $\mathcal{L}$ и \textit{нулевое подпространство} $\{ 0 \}$. Эти линейные подпространства называют \textit{несобственными}, в то время как все остальные линейные подпространства называют \textit{собственными}.

\begin{definition}[Нулевое подпространство]
  \textit{Нулевым подпространством} называется подпространство, состоящее из единственного элелемента -- нулевого.
\end{definition}

\begin{definition}[Несобственные пространства]
  Линейные подпространства $\mathcal{L}$ и нулевое подпространство линейного пространства $\mathcal{L}$ называются \textit{несобственными} .
\end{definition}

\begin{definition}[Собственные пространства]
  Линейные подпространства линейного пространства $\mathcal{L}$ за исключением несобственных называются \textit{несобственными} .
\end{definition}

Пусть в линейном пространстве $\mathcal{L}$ задана система векторов $e_1, \ldots, e_k$. Рассмотрим множество $\mathcal{H}$ всех векторов в $\mathcal{L}$, которые могут быть представлены линейной комбинацией этих векторов. Это множество является линейным подпространством в $\mathcal{L}$.
Пусть: \[
  x = x_1 e_1 + \ldots + x_k e_k \quad y = y_1 e_1 + \ldots + y_k e_k
\] 
Тогда:
\begin{gather*}
  x + y = \left(x_1 + y_1\right) e_1 + \ldots + \left( x_k + y_k \right) e_k \in H \\
  \lambda x = \left( \lambda x_1 \right) e_1 + \ldots + \left( \lambda x_k \right) e_k \in H
\end{gather*}

Описанное линейное подпространство называют \textit{линейным подпространством}. 

\begin{definition}
  Линейной оболочкой линейного пространства $\mathcal{L}$ называется совокупность всех конечных линейных комибнаций векторов данной системы.
\end{definition}

\subsection{Ранг системы векторов}

\begin{definition}[Ранг системы векторов]
  \textit{Рангом системы векторов} в линейном пространстве называют размерность линейной оболочки этой системы векторов.
\end{definition}

\begin{theorem}
  Ранг системы векторов $a = (a_1, \ldots, a_k)$ линейного пространства $\mathcal{L}$ равен:
  \begin{enumerate}
    \item максимальному количеству линейно независимых векторов в системе a;
    \item рангу матрицы, составленной по столбцам из координат векторов $a_1,\ldots, a_k$ в каком-либо базисе линейного пространства $\mathcal{L}$.
  \end{enumerate}
\end{theorem}

\subsection{Евклидово пространство}

\begin{definition}[Евклидово пространство]
  Линейное пространство $\mathcal{E}$ называют \textit{евклидовым пространством}, если в этом пространстве задано скалярное умножение, т.е. закон или правило, согласно которому каждой паре векторов $x, y \in \mathcal{E}$ поставлено в соответствие действительное число $(x, y)$, называемое скалярным произведением. При этом выполняются следующие аксиомы скалярного умножения:
  \begin{enumerate}
    \item $(x, y) = (y, x)$;
    \item $(x + y, z) = (x, z) + (y, z)$;
    \item  $(\lambda x, y) = \lambda (x, y), \quad \lambda \in \R$ ;
    \item $(x, x) > 0$, причём $\left( x, x \right) = 0$ тогда и только тогда, когда $x = 0$.
  \end{enumerate}
\end{definition}

\begin{note}
  Т.е. евклидово пространство -- это пространство, в котором определена операция \textit{скалярного произведения}. 
\end{note}

\begin{property}[1]
   \[ (x, \lambda y) = \lambda (x, y) \] 
\end{property}

\begin{property}[2]
   \[ (x, y + z) = (x, y) + (x, z) \] 
\end{property}

\begin{property}[3]
  \[ (x, 0) = 0 \] 
\end{property}

\subsection{Неравенство Коши -- Буняковского}

\begin{theorem}
  Для любых векторов $x, y$ евклидова пространства $\mathcal{E}$ справедливо неравенство:  \[
    \left( x, y \right)^2 \le (x, y)(y, y)
  \] 
\end{theorem}

\begin{definition}[Угол между векторами]
  \textit{Углом $\varphi$ между} ненулевыми \textit{векторами} $x$ и $y$ в евклидовом пространстве $\mathcal{E}$ называют такое значение  $\varphi \in (0, \pi)$ что: \[
    \cos \varphi = \frac{(x, y)}{\|x\| \|y\|}
  \] 
  где $\|x\| = \sqrt{(x, x)}$, а $\|y\| = \sqrt{(y, y)}$
\end{definition}

\subsection{Норма вектора}

\begin{definition}
  Функцию, заданную на линейном пространстве $\mathcal{L}$, которая каждому вектору $x \in \mathcal{L}$ ставит в соответствие действительное число $\|x\|$, называют \textit{нормой}, если она удовлетворяет следующим аксиомам нормы:
  \begin{enumerate}
    \item $\|x\| > 0$, причем равенство $\|x\| = 0$ возможно только при $x = 0$;
    \item $\|\lambda x\| = |\lambda| \|x\|, \lambda \in \R$;
    \item $\|x + y\| \le \|x\| + \|y\|$ (неравенство треугольника).
  \end{enumerate}
\end{definition}

\begin{theorem}
  Всякое скалярное умножение в евклидовом пространстве определяет норму согласно формуле \[
  \|x\| = \sqrt{\left( x, x \right) } 
  \] 
\end{theorem}

\subsection{Ортогональные системы векторов}

\begin{definition}
  Два вектора в евклидовом пространстве называют \textit{ортогональными}, если их скалярное произведение равно нулю. \[
  x \perp y \iff (x, y) = 0
  \] 
\end{definition}

Говорят, что вектор $x$ в евклидовом пространстве $\mathcal{E}$ ортогонален подпространству \mathcal{H}, и обозначают $x \perp \mathcal{H}$, если он ортогонален каждому вектору этого подпространства.

\begin{definition}[Ортогональная система вектором]
  Систему векторов евклидова пространства называют \textit{ортогональной}, если любые два вектора из этой системы ортогональны.
\end{definition}

\begin{theorem}
  Любая ортогональная система ненулевых вектором всегда линейно независима.
\end{theorem}

\begin{definition}[Ортогональный базис]
  Если базис евклидова пространства представляет собой ортогональную систему векторов,
  то этот базис называют \textit{ортогональным}.
\end{definition}

\begin{definition}
  Ортогональный базис называют \textit{ортонормированным}, если каждый
вектор этого базиса имеет норму, равную единице.
\end{definition}

\begin{theorem}[Теорема Пифагора]
  Если векторы $x$ и $y$ из евклидова пространства ортогональны, то: \[
    \|x + y\|s^2 = \|x\|^2 + \|y\|^2
  \] 
\end{theorem}

\begin{definition}
  Ортогональный базис называют ортонормированным, если каждый вектор этого базиса имеет норму (длину), равную единице.
\end{definition}

