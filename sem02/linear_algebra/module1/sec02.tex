\section{Линейные подпространства}

\begin{definition}[Линейное подпространства]\label{def:21}
  Подмножество $\mathbb{H}$ линейного пространства $\mathbb{L}$ называют \textit{линейным подпространством}, если выполнены следующие два условия:
  \begin{enumerate}
    \item Сумма любых двух векторов из $\mathbb{H}$ принадлежит $\mathbb{H}: \vec{x}, \vec{y} \in \mathbb{H} \implies \vec{x} + \vec{y} \in \mathbb{H}$;
    \item Произведение любого вектора из $\mathbb{H}$ на любое действительное число снова принадлежит $\mathbb{H}: \vec{x} \in \mathbb{H}, \lambda \in \R \implies \lambda \vec{x} \in \mathbb{H}$.
  \end{enumerate}
\end{definition}

Определение \ref{def:21} фактически говорит о том, что линейное подпространство -- это любое подмножество данного линейного пространства, замкнутое относительно линейных операций, т.е. применение линейных операций к векторам, принадлежащим этому подмножеству, не выводит результат за пределы подмножества.

В любом линейном пространстве $\mathbb{L}$ всегда имеются два линейных подпространства: само линейное пространство $\mathbb{L}$ и \textit{нулевое подпространство} $\{ 0 \}$. Эти линейные подпространства называют \textit{несобственными}, в то время как все остальные линейные подпространства называют \textit{собственными}.

\begin{definition}[Нулевое подпространство]
  \textit{Нулевым подпространством} называется подпространство, состоящее из единственного элелемента -- нулевого.
\end{definition}

\begin{definition}[Несобственные пространства]
  Линейные подпространства $\mathbb{L}$ и нулевое подпространство линейного пространства $\mathbb{L}$ называются \textit{несобственными} .
\end{definition}

\begin{definition}[Собственные пространства]
  Линейные подпространства линейного пространства $\mathbb{L}$ за исключением несобственных называются \textit{несобственными} .
\end{definition}

Пусть в линейном пространстве $\mathbb{L}$ задана система векторов $\vec{e}_1, \ldots, \vec{e}_k$. Рассмотрим множество $\mathbb{H}$ всех векторов в $\mathbb{L}$, которые могут быть представлены линейной комбинацией этих векторов. Это множество является линейным подпространством в $\mathbb{L}$.
Пусть: \[
  \vec{x} = \vec{x}_1 \vec{e}_1 + \ldots + \vec{x}_k \vec{e}_k \quad \vec{y} = \vec{y}_1 \vec{e}_1 + \ldots + \vec{y}_k \vec{e}_k
\] 
Тогда:
\begin{gather*}
  \vec{x} + \vec{y} = \left(\vec{x}_1 + \vec{y}_1\right) \vec{e}_1 + \ldots + \left( \vec{x}_k + \vec{y}_k \right) \vec{e}_k \in H \\
  \lambda \vec{x} = \left( \lambda \vec{x}_1 \right) \vec{e}_1 + \ldots + \left( \lambda \vec{x}_k \right) \vec{e}_k \in H
\end{gather*}

Описанное линейное подпространство называют \textit{линейным подпространством}. 

\begin{definition}
  Линейной оболочкой линейного пространства $\mathbb{L}$ называется совокупность всех конечных линейных комибнаций векторов данной системы.
\end{definition}

\subsection{Ранг системы векторов}

\begin{definition}[Ранг системы векторов]
  \textit{Рангом системы векторов} в линейном пространстве называют размерность линейной оболочки этой системы векторов.
\end{definition}

\begin{theorem}
  Ранг системы векторов $a = (a_1, \ldots, a_k)$ линейного пространства $\mathbb{L}$ равен:
  \begin{enumerate}
    \item максимальному количеству линейно независимых векторов в системе a;
    \item рангу матрицы, составленной по столбцам из координат векторов $a_1,\ldots, a_k$ в каком-либо базисе линейного пространства $\mathbb{L}$.
  \end{enumerate}
\end{theorem}

\subsection{Евклидово пространство}

\begin{definition}[Евклидово пространство]
  Линейное пространство $\mathbb{E}$ называют \textit{евклидовым пространством}, если в этом пространстве задано скалярное умножение, т.е. закон или правило, согласно которому каждой паре векторов $\vec{x}, \vec{y} \in \mathbb{E}$ поставлено в соответствие действительное число $(\vec{x}, \vec{y})$, называемое скалярным произведением. При этом выполняются следующие аксиомы скалярного умножения:
  \begin{enumerate}
    \item $(\vec{x}, \vec{y}) = (\vec{y}, \vec{x})$;
    \item $(\vec{x} + y, \vec{z}) = (\vec{x}, \vec{z}) + (\vec{y}, \vec{z})$;
    \item  $(\lambda x, \vec{y}) = \lambda (\vec{x}, \vec{y}), \quad \lambda \in \R$ ;
    \item $(\vec{x}, \vec{x}) > 0$, причём $\left( x, \vec{x} \right) = 0$ тогда и только тогда, когда $\vec{x} = \vec{0}$.
  \end{enumerate}
\end{definition}

\begin{note}
  Т.е. евклидово пространство -- это пространство, в котором определена операция \textit{скалярного произведения}. 
\end{note}

\begin{property}[1]
   \[ (\vec{x}, \lambda \vec{y}) = \lambda (\vec{x}, \vec{y}) \] 
\end{property}

\begin{property}[2]
   \[ (\vec{x}, \vec{y} + \vec{z}) = (\vec{x}, \vec{y}) + (\vec{x}, \vec{z}) \] 
\end{property}

\begin{property}[3]
  \[ (\vec{x}, \vec{0}) = 0 \] 
\end{property}

\subsection{Неравенство Коши -- Буняковского}

\begin{theorem}
  Для любых векторов $\vec{x}, \vec{y}$ евклидова пространства $\mathbb{E}$ справедливо неравенство:  \[
    \left( \vec{x}, \vec{y} \right)^2 \le (\vec{x}, \vec{y})(\vec{y}, \vec{y})
  \] 
\end{theorem}

\begin{definition}[Угол между векторами]
  \textit{Углом $\varphi$ между} ненулевыми \textit{векторами} $\vec{x}$ и $\vec{y}$ в евклидовом пространстве $\mathbb{E}$ называют такое значение  $\varphi \in (0, \pi)$ что: \[
    \cos \varphi = \frac{(x, y)}{\|x\| \|y\|}
  \] 
  где $\|x\| = \sqrt{(x, x)}$, а $\|y\| = \sqrt{(y, y)}$
\end{definition}

\subsection{Норма вектора}

\begin{definition}
  Функцию, заданную на линейном пространстве $\mathbb{L}$, которая каждому вектору $\vec{x} \in \mathbb{L}$ ставит в соответствие действительное число $\|\vec{x}\|$, называют \textit{нормой}, если она удовлетворяет следующим аксиомам нормы:
  \begin{enumerate}
    \item $\|\vec{x}\| > 0$, причем равенство $\|\vec{x}\| = 0$ возможно только при $\vec{x} = \vec{0}$;
    \item $\|\lambda \vec{x}\| = |\lambda| \|\vec{x}\|, \lambda \in \R$;
    \item $\|\vec{x} + \vec{y}\| \le \|\vec{x}\| + \|\vec{y}\|$ (неравенство треугольника).
  \end{enumerate}
\end{definition}

\begin{theorem}
  Всякое скалярное умножение в евклидовом пространстве определяет норму согласно формуле \[
  \|x\| = \sqrt{\left( \vec{x}, \vec{x} \right) } 
  \] 
\end{theorem}

\subsection{Ортогональные системы векторов}

\begin{definition}
  Два вектора в евклидовом пространстве называют \textit{ортогональными}, если их скалярное произведение равно нулю. \[
  \vec{x} \perp \vec{y} \iff (\vec{x}, \vec{y}) = 0
  \] 
\end{definition}

Говорят, что вектор $\vec{x}$ в евклидовом пространстве $\mathbb{E}$ ортогонален подпространству \mathbb{H}, и обозначают $\vec{x} \perp \mathbb{H}$, если он ортогонален каждому вектору этого подпространства.

\begin{definition}[Ортогональная система вектором]
  Систему векторов евклидова пространства называют \textit{ортогональной}, если любые два вектора из этой системы ортогональны.
\end{definition}

\begin{theorem}
  Любая ортогональная система ненулевых вектором всегда линейно независима.
\end{theorem}

\begin{definition}[Ортогональный базис]
  Если базис евклидова пространства представляет собой ортогональную систему векторов,
  то этот базис называют \textit{ортогональным}.
\end{definition}

\begin{definition}
  Ортогональный базис называют \textit{ортонормированным}, если каждый
вектор этого базиса имеет норму, равную единице.
\end{definition}

