\section{Правильные и неправильные рациональные дроби}

\begin{definition}[Рациональная дробь]
  \textit{Дробно-рациональной функцией} или \textit{рациональной дробью} называется функция, равная частному от деления двух многочленов. \[
    \frac{P_m(x)}{Q_n(x)} = \frac{a_m x^m + a_{m-1} x^{m-1} + \ldots + a_1 x + a_0}{b_n x^n + b_{n-1} x^{n-1} + \ldots + b_1 x + b_0}
  \] 
\end{definition}

\begin{definition}[Правильная рациональная дробь]
  Рациональная дробь называется \textit{правильной}, если степень числителя меньше степени знаменателя, т.е. $m < n$. 
\end{definition}

\begin{definition}[Неправильная рациональная дробь]
  Рациональная дробь называется \textit{неправильной}, если степень числителя \textit{не меньше} степени знаменателя, т.е. $m \ge n$.
\end{definition}

\subsection{Интегрирование простейших рациональных дробей}

\begin{enumerate}
  \item \[
  \int \frac{A}{x - a} dx = A \int \frac{dx}{x - a} = A \int \frac{d(x - a}{x - a} = A \ln |x - a| + C \quad \forall C
  \] 

  \item
    \begin{align*}
      \int \frac{A}{(x - a)^k} &= A \int \frac{dx}{(x - a)^k} \\
        &= A \int \frac{d(x - a)}{(x - a)^k} = A \int (x - a)^{-k} d(x - a) \\
        &= \frac{A (x - a)^{1 - k}}{1 - k} + C \quad \forall c
    \end{align*}

  \item \[
      \int \frac{Mx + N}{x^2 + px + q} dx = (*)
  \] 
    \begin{gather*}
      x^2 + px + q = x^2 + 2\cdot \frac{p}{2}x + \frac{p^2}{4} - \frac{p^2}{4} + q = \\
      = \left( x + \frac{p}{2} \right) ^2 + \left( q - \frac{p^2}{4} \right) ^2 = \left( x + \frac{p}{2} \right) ^2 + b^2
    \end{gather*}
    \begin{align*}
      (*) &= \int \frac{Mx + N}{\left( x + \frac{p}{2} \right) + b^2} dx = 
      \begin{vmatrix} 
        x + \frac{p}{2} = t \\
        x = t - \frac{p}{2} \\
        dx = dt
      \end{vmatrix} \\ 
          &= \int \frac{M(t - \frac{p}{2}) + N}{t^2 + b^2} dt = \\
          &= M \int \frac{t}{t^2 + b^2}dt + \left(N - \frac{p}{2 M}\right) \int \frac{dt}{t^2 + b^2} = \\
          &= \frac{M}{2} \int \frac{d\left( t^2 + b^2 \right)}{t^2 + b^2} + \left( N - \frac{p}{2} M \right) \frac{1}{b} \arctg \frac{t}{b} = \\
          &= \frac{M}{2} \ln|t^2  b^2| + \frac{\left( N - \frac{p}{2} M \right) }{b} \arctg \frac{t}{b} + C = \\
          &= \frac{M}{2} \ln |x^2 + px + q| + \frac{\left( N - \frac{p}{2} M \right) }{\sqrt{q - \frac{p^2}{4}} } \arctg \frac{x + \frac{p}{2}}{\sqrt{q - \frac{ps^2}{4}} } + C
    \end{align*}
\end{enumerate}

\subsection{Неправильные рациональные дроби}

Любая неправильная рациональная дробь может быть представлена в виде суммы многочлена и правильной рациональной дроби: \[
  \frac{P(x)}{Q(x)} = L(x) + \frac{r(x)}{Q(x)}
\]

Где $L(x)$ -- многочлен, частное от деления $P(x)$ на $Q(x)$; $r(x)$ -- остаток от деления $P(x)$ на $Q(x)$; $\frac{r(x)}{Q(x)}$ -- правильная рациональная дробь.

Интегрируя выражение выше мы получаем: \[
\int \frac{P(x)}{Q(x)} dx = \int L(x) dx + \int \frac{r(x)}{Q(x)}dx
\] 

Вывод: интегрирование непрерывной рациональной дроби сводится к интегрированию многочлена и правильной рациональной дроби.

\begin{theorem}[О разложении правильной рациональной дроби на простейшие]
  Любая правильная рациональная дробь $\frac{P(x)}{Q(x)}$, знаменатель которой может быть представлен в виде: \[
  Q(x) = (x - x_1)^k \cdot (x - x_2)^{k_2} \cdot \ldots \cdot (x - x_{n})^{k_n} \cdot (x^2 + p_1 x + q_1)^{s_1} \cdot \ldots \cdot (x^2 + p_m x + q_m)^{s_m}
  \] 
  может быть представлена и притом единственном образом в виде суммы многочлена и простейших дробей: \[
    \frac{P(x)}{Q(x)}= G(x) + \sum_{i=1}^{n} \frac{P_i(x)}{Q_i(x)}
  \]
\end{theorem}

\subsubsection{Метод неопределенных коэффициентов}

Равенство в теореме о разложении правильной рациональной дроби на простые представляет собой тождество. Поэтому, приводя дроби к общему знаменателю, получим тождество числителей слева и справа. Приравнивая коэффициенты при одинаковых степенях $x$ получим СЛАУ для определения неизвестных коэффициентов.

\subsubsection{Метод конкретных значений}

После получения тождества числителей, подставляем конкретные значения переменной $x$, т.к. оно верно $\forall x$. Обычно вместо $x$ подставляют действительные корни знаменателя.

