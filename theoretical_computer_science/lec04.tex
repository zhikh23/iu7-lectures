\subsection{Типы информации}

\begin{enumerate}
  \item Числа:
    \begin{enumerate}
      \item Целые:
        \begin{enumerate}
          \item Беззнаковые
          \item Знаковые
        \end{enumerate}
      \item Вещественные:
        \begin{enumerate}
          \item С фиксированной запятой
          \item С плавающей запятой
        \end{enumerate}
    \end{enumerate}
  \item Символьная 
  \item Графическая
\end{enumerate}

\subsubsection{Представление целых чисел}

Целые числа:
\begin{enumerate}
  \item Беззнаковые
  \item Знаковые
\end{enumerate}

Память имеет байтовую структуру. Адресуемая ячейка составляет несколько байтов.

\begin{definition}
  Комбинация связанных байтов, обрабатываемая совместно, называется \textbf{машинным словом}.
\end{definition}

\subsubsection{Беззнаковые целые числа}

При двоичной машине и k-разрядной сеткке наибольшее целое число: \[
  z_{max} = z^{k} - 1
\] 

\subsubsection{Знаковые целые числа}

Два варианта:
\begin{enumerate}
  \item Старший разряд считается знаковым
  \item В дополнительном коде
\end{enumerate}

Будем считать числа:
\begin{itemize}
  \item от 0 до 32737 -- положительными 
  \item от 32768 до 65535 -- отрицательными
\end{itemize}

Таким образом, судить о величине знака числа можно будет по его величине.

\begin{definition}
  \textbf{Дополнением} k-разрядного целого числа z в системе счисления p называется $D(z_p, k) = p^{k} - z$. \[
  D(z_p, k) = (p^{k} - 1) - z + 1
  \] 
\end{definition}

\subsubsection*{Дополнительный код целого числа со знаком}

Дополнительный код фоормируется в 2 этапа:
\begin{enumerate}
  \item Строится инвертированное представление исходного числа
  \item К полученной инверсии числа прибавляется 1
\end{enumerate}

Т.е. дополнительный код:
\begin{itemize}
  \item Для положительных чисел совпадает с самим числом,
  \item Для отрицательных чисел совпадает с дополнением модуля исходного числа.
\end{itemize}

\subsubsection*{Машинная арифметика целых чисел. Пример}

\begin{gather*}
  27_{10} = 00011011_2 \\
  -3_{10} = D(3_{10}, 8) \\
  3_{10} = 00000011_2 \\
  (-3_{10}) = 11111101
\end{gather*}

Тогда:
\begin{gather*}
  \begin{table}[htpb]
    \centering
    \label{tab:label}
    \begin{tabular}{c}
      0.00011011 \\
      0.11111101 \\
      \hline
      1.00111000
    \end{tabular}
  \end{table}
\end{gather*}

\subsubsection*{Машинная арифметика целых чисел. Пример 2}

\begin{gather*}
  3_{10} = 00000011_2 \\
  -21_{10} = D(21_{10}, 8) \\
  21_{10} = 00011011_2 \\
  (-3_{10}) = 11100101
\end{gather*}

Тогда:
\begin{gather*}
  \begin{table}[htpb]
    \centering
    \label{tab:label}
    \begin{tabular}{c}
      0.00000011 \\
      0.11100101 \\
      \hline
      1.11101000
    \end{tabular}
  \end{table}
\end{gather*}

\subsubsection{Представление вещественных чисел}

В ЭВМ для записи чисел отводится конечное число разрядов.

На числовой оси вещественных числа образуют непрерывное множество.

Таким образом, строгое отношение между числами превращается в нестрогое для их представлений. \[
x_1 < x_2 \implies x_1` \le x_2`
\] 

\begin{note}
  Арифметика над вещественными числами в компьютере всегда ведётся с погрешностью! 
\end{note}

\begin{definition}
  \textbf{Машинный ноль} -- некое малое число, все числа, меньше которого принимаются равными нулю.
\end{definition}

\subsubsection{Формы представления вещественныъ чисел}

\begin{itemize}
  \item С фиксированной запятой
  \item С плавающей запятой
    \begin{itemize}
      \item Нормальная форма
      \item Нормализованная форма
    \end{itemize}
\end{itemize}

\subsubsection*{Представление числа с плавающей запятой}

\begin{gather*}
  X = +- M \cdot P^{+-r}
\end{gather*}

M - мантисса числа (в пределах от 0 до 1)
P - основание системы счисления
r - порядок числа

