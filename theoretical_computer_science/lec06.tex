\section{Логические основы информатики}

Создатель алгебры логики -- \textit{Джордж Буль}.

\begin{definition}
  \textbf{Высказывание} -- определённое суждение, о котором можно утверждать, истинно оно или ложно.
\end{definition}

\begin{definition}
  \textbf{Логическая переменная} -- это величина, которая принимает только одно из двух значений: 0 или 1.
\end{definition}

Высказывание \textit{абсолютно истинно}, если соответствующая ей логическая величина принимает значение True при любых условиях.
Высказывание \textit{абсолютно ложным}, если соответствующая ей логическая величина принимает значение False при любых условиях.

\begin{definition}
  \textbf{Логическая функция} -- функция $f(x_1, \ldots x_{n})$, принимающая значение 0 или 1 на наборе переменных.
\end{definition}

\subsection{Основные логические операции}

\begin{enumerate}
  \item Логическое умножение (логическое И, конъюнкция) \[
    A \cdot B, A and B, A \& B
  \] 

  \item Логическое сложение (логическое ИЛИ, дизъюнкция) \[
    A+B, A or B, A | B
\]

  \item Логическое отрицание (логическое НЕ) \[
    not A, \sim A
\] 
\end{enumerate}

Порядок действий в логических операциях:
\begin{enumerate}
  \item Отрицание
  \item Умножение
  \item Сложение
\end{enumerate}

\subsection*{Основные логические функции}

\begin{enumerate}
  \item дизъюнкция 
  \item конъюнкция 
  \item сложение по модулю два (исключающее или)
  \item импликация
  \item эквивалентность 
  \item функция Шеффера
  \item функция Пирса (стрелка Пирса)
\end{enumerate}

% Тут будут таблицы для всех функций:
% дизъюнкция
% конъюнкция
% исключающее или
% эквивалентность
% импликация

\begin{definition}
  Набор функций, которые методом суперпозиции обеспечивает представление любой друго й функции, называют \textbf{функционально полным набором}.
\end{definition}

\begin{definition}
  \textbf{ОФПН (основной функционально полный набор} -- набор, базирующийся на операциях конъюнкции, дизъюнкции и отрицания.
\end{definition}

Существуют и другие базисы.

\begin{definition}
  Функции будут \textbf{равносильны} друг другу, если они на любых наборах переменных принимают одни и те же значения.
\end{definition}

\subsubsection{Построение таблиц истинности по логическому выражению}

Если провести вычисления значений логического выражения для всех комбинаций входных значений, то получаем таблицу истинности.

\begin{eg}
Пусть дана функция:
\[
f(x_1, x_2, x_3) = (x_1 + \sim x_2)(\sim x_1 + x_3) + x_2 x_3
\] 
\end{eg}
\begin{table}[htpb]
  \centering
  \caption{Пример таблицы истинности}
  \label{tab:label}
  \begin{tabular}{ c | c  | c | c }
    x_1 & x_2 & x_3 & f(x_1, x_2, x_3) \\
    \hline 
    0 & 0 & 0 & 1 \\
    0 & 0 & 1 & 1 \\
    0 & 1 & 0 & 0 \\
    0 & 1 & 1 & 1 \\
    1 & 0 & 0 & 0 \\
    1 & 0 & 1 & 1 \\
    1 & 1 & 0 & 0 \\
    1 & 1 & 1 & 1 \\
  \end{tabular}
\end{table}

\subsubsection{Основные логические схемы компьютера}

% Вставить картиночки AND, OR, NOT, NAND, NOR

