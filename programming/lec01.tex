\lesson{1}{Программирование}

\section{Компьютер. Языки программирования}

\begin{definition}
  Компьютер - устройство, спосбоное выполнять заданную, четко определенную, изменяемую последовательность операций.
\end{definition}

\begin{definition}
  \textbf{Электронно вычислительная машина (ЭВМ)} - основной вид реализации компьютеров, который технически выполнен на электронных элементах.
\end{definition}

\subsection{Схема ЭВМ}

Устройство ввода $\implies$ Процессор и Память $\implies$ Устройство вывода

\subsection{Элементарные термины}

\begin{definition}
  Процессор - интегральная схема, использующая машинные инструкции (код программы), главная часть аппаратного обеспечения компьютера.
\end{definition}

\begin{definition}
  Машинный код - система команд (набор готовых инструкций) конкретной вычислительной машины, которая интепретируется непосредственно процессором. Кодируется в двоичном видe.
\end{definition}

\begin{definition}
  Файл - наименованное место на диске.
\end{definition}

\begin{definition}
  Алгоритм - конечная совокупность точно заданных правил решения некоторого класса задач или набор инструкций, описывающих порядок действий исполнителя для решения определенной задачи.
\end{definition}

\begin{definition}
  Испольняемая программа - сочетание компьютерных инструкций и данных, позволяющее аппаратному обеспечению вычислительной системы выполнять вычисления или функции управления.
\end{definition}

\begin{definition}
  Язык программирования - формальный язык, предназначенный для записи компьютерных программ. Яызк программирования определяет набор лексических, синтексических и семантических правил, определяющих действия, которые выполнит ЭВМ под ее управлением.
\end{definition}

\subsection{Способы классификации ЯП}
\begin{itemize}
  \item Уровень абстракции от аппаратной части:
    \begin{enumerate}
      \item низкоуровневые
      \item высокоуровневые
    \end{enumerate}
  \item Способ выполнения готовой программы:
    \begin{enumerate}
      \item компилируемые
      \item интепретируемые
    \end{enumerate}
  \item Используемая парадигма:
    \begin{enumerate}
      \item императивные (процедурные) языки
      \item аппликативные (функциональные) языки
      \item языки системы правил (декларативные) языки
      \item объектно-ориентированные языки
    \end{enumerate}
\end{itemize}

