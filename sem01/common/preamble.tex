\usepackage[utf8]{inputenc}
\usepackage[russian]{babel}
\usepackage{cancel}
\usepackage[none]{hyphenat}

% IGNORE USELESS WARNINGS ===========================================
\usepackage{silence}
\WarningFilter{mdframed}{You got a bad break}

% HEADER AND FOOTER =================================================
\def\@header{}
\newcommand{\lectures}[1]{
  \def\@header{#1. Лекции}
  \begin{center}
    \section*{#1. Лекции}
  \end{center}
}

\usepackage{fancyhdr}
\pagestyle{fancy}
\setlength{\headheight}{22.54279pt}

\fancyhead[R]{\@header}
\fancyhead[R]{}
\fancyfoot[L]{\thepage}
\fancyfoot[C]{\leftmark}

\makeatother

% THEOREMS ==========================================================
\usepackage{amsmath, amsfonts, mathtools, amsthm, amssymb}
\usepackage{thmtools}
\usepackage[usenames,dvipsnames]{xcolor}

\declaretheoremstyle[
    headfont=\bfseries\sffamily\color{ForestGreen!70!black}, 
    bodyfont=\normalfont,
    mdframed={
        linewidth=2pt,
        rightline=false, topline=false, bottomline=false,
        linecolor=ForestGreen, 
    }
]{thmgreenbox}

\declaretheoremstyle[
    headfont=\bfseries\sffamily\color{NavyBlue!70!black}, 
    bodyfont=\normalfont,
    mdframed={
        linewidth=2pt,
        rightline=false, topline=false, bottomline=false,
        linecolor=NavyBlue, 
    }
]{thmbluebox}

\declaretheoremstyle[
    headfont=\bfseries\sffamily\color{NavyBlue!70!black}, 
    bodyfont=\normalfont,
    mdframed={
        linewidth=2pt,
        rightline=false, topline=false, bottomline=false,
        linecolor=NavyBlue
    }
]{thmblueline}

\declaretheoremstyle[
    headfont=\bfseries\sffamily\color{RawSienna!70!black}, 
    bodyfont=\normalfont,
    mdframed={
        linewidth=2pt,
        rightline=false, topline=false, bottomline=false,
        linecolor=RawSienna, 
    }
]{thmredbox}

\declaretheoremstyle[
    headfont=\bfseries\sffamily\color{RawSienna!70!black}, 
    bodyfont=\normalfont,
    numbered=no,
    mdframed={
        linewidth=2pt,
        rightline=false, topline=false, bottomline=false,
        linecolor=RawSienna, 
    },
    qed=\qedsymbol
]{thmproofbox}

\declaretheoremstyle[
    headfont=\bfseries\sffamily\color{NavyBlue!70!black}, 
    bodyfont=\normalfont,
    numbered=no,
    mdframed={
        linewidth=2pt,
        rightline=false, topline=false, bottomline=false,
        linecolor=NavyBlue, 
    },
]{thmexplanationbox}

\theoremstyle{plain}

\declaretheorem[style=thmgreenbox, name=Определение, parent=section]{definition}
\declaretheorem[style=thmbluebox, name=Вывод, numbered=no]{corollary}

\declaretheorem[style=thmredbox, name=Теорема, parent=section]{theorem}
\declaretheorem[style=thmproofbox, name=Доказательство]{replacementproof}
\renewenvironment{proof}[1][\proofname]{\vspace{-10pt}\begin{replacementproof}}{\end{replacementproof}}

\declaretheorem[style=thmbluebox, numbered=no, name=Пример]{eg}
\declaretheorem[style=thmblueline, numbered=no, name=Замечание]{note}


% MATH SYMBOLS ======================================================
\newcommand\N{\ensuremath{\mathbb{N}}}
\newcommand\R{\ensuremath{\mathbb{R}}}
\newcommand\Z{\ensuremath{\mathbb{Z}}}
\renewcommand\O{\ensuremath{\emptyset}}
\newcommand\Q{\ensuremath{\mathbb{Q}}}
