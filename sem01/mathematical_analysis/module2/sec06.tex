\subsection{Лекция 13.12.23}

4. $f(x) = (1+x)^\alpha$, $\alpha \in \R$.
\begin{gather*}
  f'(x) = \alpha(1+x)^{\alpha-1} \\
  f''(x) = \alpha(\alpha - 1)(1+x)^{\alpha-2} \\
  f'''(x) = \alpha(\alpha - 1)(\alpha - 2)(1+x)^{\alpha-3} \\
  \ldots \\
  f^{(n)} = \alpha(\alpha - 1)\ldots(\alpha - (n - 1))(1 + x)^{\alpha - n} \\
  f^{(n+1)} = \alpha(\alpha - 1)\ldots(\alpha - n)(1 + x)^{\alpha - (n + 1)} \\
\end{gather*}
\begin{gather*}
  f(0) = 1 \\
  f'(0) = \alpha \\
  f''(0) = \alpha (\alpha - 1) \\
  f'''(0) = \alpha (\alpha - 1)(\alpha - 2) \\
  \ldots \\
  f^{(n)} = \alpha(\alpha - 1) \ldots (\alpha - (n-1))
\end{gather*}
\begin{gather*}
  f(x) = f(0) + \frac{f'(0)}{1!}x + \frac{f''(x)}{2!}x^2 + \ldots + \frac{f^{(n)}(0)}{n!}x^{n} + R_n(x) \\
  (1 + x)^{\alpha} = 1 + \frac{\alpha}{1!}x + \frac{\alpha(\alpha - 1)}{2!}x^2 + \ldots + \frac{\alpha(\alpha - 1)\ldots (\alpha - (n-1))}{n!}x^{n} + R_n(x) \\
  R_n(x) = o(x^n) \\
  R_n(x) = \{ \frac{f^{(n+1)} \theta x}{(x + 1)!}x^{(n+1)} \} \\
    = \frac{\alpha(\alpha - 1) \ldots (\alpha - n)}{(n+1)!} (1 + \theta x)^{\alpha + 1}x^{n+1} \\
\end{gather*}

5. $y = f(x) = \ln(1 + x)$
\begin{gather*}
  f'(x) = \frac{1}{1+x} = (1 + x)^{-1} \\
  f''(x) = -1 \cdot \frac{1}{(1+x)^2} = -1 \cdot (1 + x)^{-2} \\
  f'''(x) = -1 \cdot (-2) \cdot \frac{1}{(1+x)^3} = -1 \cdot (-2) \cdot (1 + x)^{-3} \\
  \ldots \\
  f^{(n)} = (-1)^{n+1}(n-1)!(1-x)^{-n} 
\end{gather*}
\begin{gather*}
  f'(0) = 1 = 0! \\
  f''(0) = -1 = (-1) 1! \\
  f'''(0) = 2 = 2! \\
  f''(0) = (-1) 3! \\
  \ldots \\
  f^{(n)} = (-1)^{n+1}(n-1)!(1+x)^{-n}
\end{gather*}
\begin{gather*}
  \ln(1+x) = 0 + \frac{0!}{1!} x - \frac{1}{2!} x^2 + \frac{2!}{3!} x^{3} - \frac{3!}{4!} x^{4} + \ldots + \frac{(-1)^{n+1}(n-1)!}{n!} + R_n(x) \\
  \ln(1+x) = \frac{1}{1}x - \frac{1}{2}x^2 + \frac{1}{3}x^{3} - \ldots + \frac{(-1)^{n+1}}{n}x^n + R_n(x) \\
  R_n(x) = o(x^{n}) \\
  R_n(x) = \{ \frac{f^{(n+1)}(\theta x)}{(n+1)!}x^{n+1} \} = \\
  = \frac{(-1)^{n+1}n! (1 + \theta x)^{-(n+1)}}{(n+1)!}x^{n+1} = \\
  = \frac{(-1)^{n+2}(1+\theta x)^{-(n+1)}}{n+1} x^{n+1}
\end{gather*}

\section{Вертикальные, наклонные, горизонтальные ассимптоты}

\begin{definition}
  \textit{Ассимптотой} графика функции $y = f(x)$ называется прямая расстояние до которой от точки, лежащей на графике, стремится к нулю при удалении от начала координат.
\end{definition}

\begin{definition}
  Прямая $x = a$ называется \textit{вертикальной ассимптотой} графика функции $y = f(x)$, если хотя бы один из пределов $\lim_{x \to a+} f(x)$, $\lim_{x \to a-} f(x)$ равен $\infty$.
\end{definition}
\begin{eg}
  \begin{gather*}
    y = \frac{1}{x - a} \\
    \lim_{x \to a-} \frac{1}{x-a} = -\infty \\
    \lim_{x \to a+} \frac{1}{x-a} = +\infty
  \end{gather*}
  $x = a$ -- вертикальная ассимптота.
\end{eg}
\begin{eg}
  \begin{gather*}
    y = \ln x, \quad D_f = (0, +\infty) \\
    \lim_{x \to 0+} \ln x = -\infty 
  \end{gather*}
  $x = 0$ -- вертикальная ассимптота правая.
\end{eg}

Вывод: вертикальные ассимптоты ищем среди точек разрыва функции и граничных точек.

\begin{definition}
  Прямая $y = kx + b$ называется наклонной ассимптотой графика функции  $y=f(x)$ при $x \to  \pm\infty$, если сама функция представима в виде $f(x) = kx + b + \alpha(x)$, где $\alpha(x)$ -- б.м.ф при $x \to \pm\infty$.
\end{definition}

\begin{theorem}
  \textit{Необходимое и достаточной условие существования наклонной ассимптоты}. \\ 
   График функции $y = f(x)$ имеет при  $x \to \pm\infty$ наклонную ассимптоту тогда и только тогда, когда существуют два конечных предела:
   \begin{align*}
     \begin{cases}
        &\lim_{x \to \pm\infty} \frac{f(x)}{x} \\
        &\lim_{x \to \pm\infty} (f(x) - kx)
     \end{cases} \tag{*} 
   \end{align*}
\end{theorem}
\begin{proof}
  \textit{Необходимость}. \\
  Дано $y = kx + b$ наклонная ассимптота. Доказать  $\exists $ пределы (*).\\
  По условию $y = kx + b$ -- наклонная ассимптота  $\implies $ по определению $f(x) = kx + b + \alpha(x)$, где $\alpha(x)$ -- б.м.ф. при $x \to  \pm \infty$.
  Рассмотрим:
  \begin{align*}
    \lim_{x \to \pm\infty} \frac{f(x)}{x} &= \lim_{x \to \pm\infty} \frac{kx + b + \alpha(x)}{x} = \\
    &= \lim_{x \to \pm\infty} (k + b \cdot \frac{1}{x} + \frac{1}{x} \alpha(x)) \\
    &= k + b \lim_{x \to \pm\infty} \frac{1}{x} + \lim_{x \to \pm\infty}\frac{1}{x} \alpha(x) \\
    &= k + b\cdot 0 + 0 = k \\
  \end{align*}
  Рассмотрим выражение:
  \begin{align*}
    f(x) - kx = kx + b + \alpha(x) - kx = b + \alpha(x) \\
    \lim_{x \to \pm\infty} (f(x) - kx)) = \lim_{x \to \pm\infty} (b + \alpha(x)) = b 
  \end{align*}

  \textit{Достаточность}. \\ 
  Дано $\exists $ конечные пределы (*). Доказать $y = kx + b$ -- наклонная ассимптота. \\

   $\exists $ конечный предел $\lim_{x \to \pm\infty} (f(x) - kx) = b$
   По теореме о связи функции, её предела и б.м.ф. $\implies$ \[
   f(x) - kx = b + \alpha(x)
   \]  при $x \to \pm\infty$. Выразим $f(x)$:  \[
   f(x) = kx + b + \alpha(x)
   \] где $\alpha(x)$ б.м.ф при $x \to \pm\infty$.
   По определению $\implies y = kx + b$ -- наклонная ассимптота к графику функции $y = f(x)$
\end{proof}

\begin{definition}
  Прямая $y = b$ нельзя горизонтальной ассимптотой графика функции  $y = f(x)$ x, если $\lim_{x \to \pm\infty} f(x) = b$.
\end{definition}
\begin{corollary}
  Горизонтальные ассимптоты являются частным случаем наклонных при $k = 0$.
\end{corollary}

\section{Исследование по первой производной}

\begin{definition}
  Функция $y = f(x)$, определённая на интервале $(a, b)$ \textit{возрастает} (\textit{убывает}) на этом интервале, если для любых  $x_1, x_2 \in (a, b)$ таких что $x_2 > x_1 \implies f(x_2) > f(x_1) \quad (f(x_2) < f(x_1))$.
\end{definition}

\begin{definition}
  Функция $y = f(x)$, определённая на интервале $(a, b)$ \textit{не убывает} (\textit{не возрастает} ) на этом интервале, если для любых  $x_1, x_2 \in (a, b)$ таких что $x_2 > x_1 \implies f(x_2) \ge f(x_1) \quad (f(x_2) \le f(x_1))$.
\end{definition}

\begin{definition}
  Возрастающая + убывающая функция -- называются \textit{строго монотонными} .
\end{definition}

\begin{definition}
  Невозрастающая + неубывающая функция -- называются \textit{монотонными}.
\end{definition}

\begin{theorem}
  \textit{Необходимое и достаточное условие невозрастания (неубывания) дифференцируемой функции}. \\
  Дифференцируемая на интервале $(a, b)$ не возрастает (не убывает) на этом интервале тогда и только тогда, когда  $f'(x) \le 0$ $\left( f'(x) \ge 0 \right)$ $\forall x \in (a, b)$.
\end{theorem}
\begin{proof}
  \textit{Необходимость}. \\
  Дано: $y=f(x)$ не возрастает на $(a, b)$. \\
  Доказать:  $\forall x \in (a, b) \quad f'(x) \le 0$. \[
  \forall x \in (a, b)
  \] 
  $\Delta x$ -- приращение аргумента
  \begin{gather*}
    x \to  x + \Delta x \\
    \Delta  y = y(x + \Delta x) - y(x)
  \end{gather*}
  -- приращение функции.

  1 случай: $\Delta x > 0$: \\
  т.к. $y = f(x)$ не возрастает на  $a, b$.  \[
  y(x + \Delta x) \le  y(x)
  \] \[
  \Delta  y = y(x + \Delta x) - y(x) \le 0.
  \] 
  Тогда $\frac{\Delta y}{\Delta x} = \left( \frac{+}{-} \right) \le 0$. \\

  2 случай: $\Delta  x < 0$: 
  т.к. $y = f(x)$ не возрастает на  $a, b$.  \[
  y(x + \Delta x) \ge y(x)
  \] \[
  \Delta  y = y(x + \Delta x) - y(x) \ge 0.
  \] 
  Тогда $\frac{\Delta y}{\Delta x} = \left( \frac{+}{-} \right) \le 0$. \\

  По теореме о предельном перехорде в неравенстве: 
  \begin{gather*}
    \lim_{\Delta x \to 0} \frac{\Delta y}{\Delta x} \le 0 
  \end{gather*}
  По определению производной $f'(x) \le 0$

  \textit{Достаточность}. \\
  Дано: $\forall x \in (a, b) \quad f'(x) \le 0$.
  Доказать: $y = f(x)$ не возрастает на  $a, b$.
   \begin{gather*}
    \forall x_1, x_2 \in (a,b) : x_2 > x_1
  \end{gather*}
  Рассмотрим $[x_1, x_2]$. Функция на отрезке $[x_1, x_2]$ удовлетворяет условиям теоремы Лагранжа:
  \begin{enumerate}
    \item Непрерывность на $[x_1, x_2]$. \\
      По условию $y = f(x)$ дифференцируема на интервале  $(a, b)$. По теореме о связи дифференцируемости и непрерывности функции  $\implies y=f(x)$ -- непрерывна на $[x_1, x_2]$.
    \item дифференцируемость на $(x_1, x_2)$ т.к. функция по условию дифференцируема на отрезке $[x_1, x_2]$.
  \end{enumerate}
  По теореме Лагранжа $\exists c \in (x_1, x_2)$: \[
  f(c) = \frac{f(x_2) - f(x_1)}{x_2 - x_1}
  \] 
  Т.к. $x_2 > x_1 \implies x_2 - x_1 > 0$. По условию $f'(x) \le  0, \forall x \in  (a, b) \implies f'(c) \le 0$. \\
  Тогда:
  \begin{gather*}
    f'(c) = \frac{f(x_2) - f(x_1)}{x_2 - x_1} \le 0 \\
    \implies f(x_2) - f(x_1) \le  0 \text{ при } x_2 > x_1 \\
    f(x_2) \le  f(x_1) \text{ при } x_2 > x_1
  \end{gather*}
  $\implies$ по определению функция $y = f(x)$ не возрастает на $(a, b)$.
\end{proof}
Для неубывающей функции:
\begin{proof}
  \textit{Необходимость}. \\
  Дано: $y=f(x)$ не убывает на $(a, b)$. \\
  Доказать:  $\forall x \in (a, b) \quad f'(x) \ge 0$. \[
  \forall x \in (a, b)
  \] 
  $\Delta x$ -- приращение аргумента
  \begin{gather*}
    x \to  x + \Delta x
    \Delta  y = y(x + \Delta x) - y(x)
  \end{gather*}
  -- приращение функции.

  1 случай: $\Delta x > 0$: \\
  т.к. $y = f(x)$ не убывает на  $a, b$.  \[
  y(x + \Delta x) \ge  y(x)
  \] \[
  \Delta  y = y(x + \Delta x) - y(x) \le 0.
  \] 
  Тогда $\frac{\Delta y}{\Delta x} = \left( \frac{-}{+} \right) \ge 0$. \\

  2 случай: $\Delta  x < 0$: 
  т.к. $y = f(x)$ не возрастает на  $a, b$.  \[
  y(x + \Delta x) \le y(x)
  \] \[
  \Delta  y = y(x + \Delta x) - y(x) \le 0.
  \] 
  Тогда $\frac{\Delta y}{\Delta x} = \left( \frac{-}{+} \right) \ge 0$. \\

  По теореме о предельном перехорде в неравенстве: 
  \begin{gather*}
    \lim_{\Delta x \to 0} \frac{\Delta y}{\Delta x} \le 0 
  \end{gather*}
  По определению производной $f'(x) \le 0$

  \textit{Достаточность}. \\
  Дано: $\forall x \in (a, b) \quad f'(x) \ge 0$.
  Доказать: $y = f(x)$ не убывает на  $a, b$.
   \begin{gather*}
    \forall x_1, x_2 \in (a,b) : x_2 > x_1
  \end{gather*}
  Рассмотрим $[x_1, x_2]$. Функция на отрезке $[x_1, x_2]$ удовлетворяет условиям теоремы Лагранжа:
  \begin{enumerate}
    \item Непрерывность на $[x_1, x_2]$. \\
      По условию $y = f(x)$ дифференцируема на интервале  $(a, b)$. По теореме о связи дифференцируемости и непрерывности функции  $\implies y=f(x)$ -- непрерывна на $[x_1, x_2]$.
    \item Дифференцируемость на $(x_1, x_2)$ т.к. функция по условию дифференцируема на отрезке $[x_1, x_2]$.
  \end{enumerate}
  По теореме Лагранжа $\exists c \in (x_1, x_2)$: \[
  f(c) = \frac{f(x_2) - f(x_1)}{x_2 - x_1}
  \] 
  Т.к. $x_2 > x_1 \implies x_2 - x_1 > 0$. По условию $f'(x) \ge  0, \forall x \in  (a, b) \implies f'(c) \ge 0$. \\
  Тогда:
  \begin{gather*}
    f'(c) = \frac{f(x_2) - f(x_1)}{x_2 - x_1} \ge 0 \\
    \implies f(x_2) - f(x_1) \ge  0 \text{ при } x_2 > x_1 \\
    f(x_2) \ge  f(x_1) \text{ при } x_2 > x_1
  \end{gather*}
  $\implies$ по определению функция $y = f(x)$ не убывает на $(a, b)$.
\end{proof}

\begin{theorem}
  \textit{Необходимое услоивие строгой монотонности}. \\
  Если дифференцируемая на интервале $(a, b)$ функция $y=f(x)$ возрастает (убывает) на это м интервале, то $\forall x \in (a, b)$ верно неравенство $f'(x) \ge 0$ ($f'(x) \le 0$).
\end{theorem}

\begin{theorem}
  \textit{Достаточное условие строгой монотонности}. \\
  Если для дифференцируемой на интервале $(a, b)$ функции  $y=f(x)$ выполнены условия:
  \begin{enumerate}
    \item $f'(x) \ge 0$ ($f'(x) \le 0)$ $\forall x \in (a, b)$.
    \item $f'(x)$ не обращается в ноль ни на каком промежутке  $I \subseteq (a, b) $, то функция $y = f(x)$ возрастает (убывает) на $(a, b)$.
  \end{enumerate}
\end{theorem}

\subsection{Экстремумы функции}

\begin{definition}
  Пусть $y=f(x)$ определана на интервале $(a, b), \quad x_0 \in (a, b)$. Тогда: 
  \begin{enumerate}
    \item Если $\exists \mathring{S}(x_0), \quad \forall x \in  \mathring{S}(x_0), \quad f(x) \le  f(x_0)$, то $x_0$ -- точка локального максимумаю $y = y(x_0)$ -- локальный максимум.
    \item Если $\exists  \mathring{S}(x_0) : \forall x \in \mathring{S}(x_0), f(x) \ge  f(x_0)$, то $x_0$ -- точка локального минимума. $y=y(x_0)$ -- локальный максимум.
  \end{enumerate}
\end{definition}

\begin{definition}
  Точки локального максимума и минимума называются \textit{точками экстремума}.
\end{definition}

\begin{definition}
  Локальный максимум и локальный минимум называются \textit{экстремума}.
\end{definition}

\begin{theorem}
  \textit{Необходимое условие существования экстремума}. \\
  Если $y=f(x)$ дифференцируема на интервале $(a, b)$ и  $x_0 \in (a, b)$ существует экстремум, то $f'(x_0) = 0$.
\end{theorem}
\begin{eg}
  \begin{gather*}
    y = x^2, \quad x_0 = 0 \\
    y' = 2x \quad y'(0) = 0
  \end{gather*}
\end{eg}
\begin{eg}
  \begin{gather*}
    y = x^{3}, \quad x_0 = 0 \text{ -- не явл. т. экстремума!} \\
    y' = 3x^2 \quad y'(0) = 0
  \end{gather*}
\end{eg}

\begin{definition}
  Точки, в которых производная функции обращается в ноль называются \textit{стационарными}. \[
  f'(x_0) = 0 \implies x_0 \text{ -- стационарная точка}
  \] 
\end{definition}

\begin{definition}
  Точки, в которых производная функции обращается в ноль или не существует, называются \textit{критическими точками первого порядка}.
\end{definition}

\begin{eg}
  \begin{gather*}
    y = |x|, x_0 = 0 \text{ -- точка минимума} \\
    \text{ но } \not\exists y'
  \end{gather*}
\end{eg}
\begin{eg}
  \begin{gather*}
    y = x^{\frac{2}{3}}, \quad x_0 = 0 \text{ -- точка минимума} \\
    y' = \frac{2}{3x^{-\frac{1}{3}}} = \frac{2}{3\sqrt{x}}, \quad \not\exists y'(x_0)
  \end{gather*}
\end{eg}

Вывод: точки экстремума могут быть двух видов:
\begin{enumerate}
  \item $f'(x) = 0$ -- гладкий экстремум.
  \item  $\not\exists f'(x)$ -- острый экстремум.
\end{enumerate}

\begin{theorem}
  \textit{Первый достаточный признак локального экстремума}. \\
  Пусть функция $y=f(x)$ непрерывна в $S(x_0)$, где $x_0$ -- критическая точка первого порядка; функция дифференцируема в $\mathring{S}(x_0)$. Тогда если проивзодная функции меняет свой знак при переходе черех точку $x_0$, то эта точка $x_0$ -- точка экстремума. Причём:
  \begin{enumerate}
    \item Если при $x < x_0$ $f'(x) > 0$, а при  $x > x_0$ $f'(x) < 0$, то  $x_0$ -- точка максимума.
    \item Если при $x < x_0$ $f'(x) < 0$, а при  $x > x_0$ $f'(x) > 0$, то  $x_0$ -- точка минимума.
  \end{enumerate}
\end{theorem}
\begin{proof}
  $\forall x \in S(x_0)$. Пусть $x > x_0$, тогда рассматриваем отрезок $[x_0, x]$. Тогда функция $y = f(x)$ удовлетворяет условиям теоремы Лагранжа:
  \begin{enumerate}
    \item Непрерывна на $[x_0, x]$, т.к. по условию функция непрерывна в $S(x_0)$, а следовательно $y=f(x)$ будет непрерывна и на меньшем промежутке $[x_0, x]$.
    \item Дифференцируема на $(x_0, x)$, т.к. по условия функция непрерывна в $\mathring{S}(x_0) \implies y = f(x)$ дифференцируема на $(x_0, x)$
  \end{enumerate}

  По теореме Лагранжа $\exists  c \in  (x_0, x)$ \[
    f'(c) = \frac{f(x) - f(x_0)}{x - x_0}
  \]
  При $x > x_0$ $x - x_0 > 0$. 
  По условию \\
  1) при $x > x_0 \quad f'(x) < 0 \quad \implies f'(c) = \frac{f(x) - f(x_0)}{x - x_0} < 0 \quad \implies f(x) < f(x_0)$ по определению строгого  $x_0$ -- точка локального максимума.
  2) при $x < x_0 \quad f'(x) > 0 \quad \implies f'(c) = \frac{f(x) - f(x_0)}{x - x_0} > 0 \quad \implies f(x) > f(x_0)$ по определению строгого  $x_0$ -- точка локального минимума.

  По теореме Лагранжа $\exists c \in (x, x_0)$: \[
  f'(c) = \frac{f(x_0) - f(x)}{x_0 - x}
  \] 
  Т.к. $x < x_0$, то $x - x_0 < 0 \quad \implies \quad x_0 - x > 0$.
  По условию \\
  1) при $x < x_0 \quad f'(x) > 0 \quad \implies f'(c) = \frac{f(x_0) - f(x)}{x_0 - x} > 0 \quad \implies f(x_0) > f(x)$ по определению строгого  $x_0$ -- точка локального максимума.
  2) при $x > x_0 \quad f'(x) > 0 \quad \implies f'(c) = \frac{f(x_0) - f(x)}{x_0 - x} <` 0 \quad \implies f(x) < f(x_0)$ по определению строгого  $x_0$ -- точка локального минимума.
\end{proof}

\begin{theorem}
  \textit{Второй достаточный признак локального экстремума}. \\
  Пусть функция $y = f(x)$ дважды дифференцируема в точке $x_0$, и $f'(x_0) = 0$. Тогда:
   \begin{enumerate}
    \item Если $f''(x_0) < 0$, то $x_0$ -- точка строго максимума. 
    \item Если $f''(x_0) > 0$, то $x_0$ -- точка строго минимума. 
  \end{enumerate}
\end{theorem}
\begin{proof}
  Разложим функцию $y = f(x)$ в окрестности точки $x_0$ по формуле Тейлора:
  \begin{gather*}
    f(x) = f(x_0) + \frac{f'(x_0)}{1!} (x - x_0) + \frac{f''(x_0)}{2!} (x - x_0)^2 + o((x - x_0)^{2})
  \end{gather*}
  Т.к. $f'(x_0) = 0$, то
  \begin{align*}
    f(x) = f(x_0) + \frac{f''(x_0)}{2!}(x - x_0)^2 + o((x - x_0)^2) \\
    f(x) - f(x_0) = \frac{f''(x_0)}{2!}(x - x_0)^2 + o((x - x_0)^2)
  \end{align*}
  Знак $f(x) - f(x_0)$ определяет $f''(x_0)$, т.к. $o((x - x_0)^2)$ -- б.м.ф. при $x \to  x_0$. 
  Если $f(x) - f(x_0) < 0$ то $f(x) < f(x_0), \quad \forall x \in  S(x_0)$
  По определению $x_0$ -- точка локального максимума.

  Если $f(x) - f(x_0) > 0$ то $f(x) < f(x_0), \quad \forall x \in  S(x_0)$
  По определению $x_0$ -- точка локального минимума.
\end{proof}

\section{Исследование по второй производной}

\begin{definition}
  Говорят, что график функции $y = f(x)$ на интервале $(a, b)$ выпуклый (выпуклый вверх) на этом интервале, если касательная к нему в любой точке этого интервала (кроме точки касания) лежит выше графика функции.
\end{definition}

\begin{definition}
  Говорят, что график функции $y = f(x)$ на интервале $(a, b)$ вогнутый (выпуклый вниз) на этом интервале, если касательная к нему в любой точке этого интервала (кроме точки касания) лежит ниже графика функции.
\end{definition}
