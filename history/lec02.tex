\section{Период политической раздробленности}

Русь распадается на большое количество княжеств (около 280). Однако данный процесс происходит и в других странах.

\subsection{Причины феодальной раздробленности}

\begin{enumerate}
  \item Лестничный порядок наследования
  \item Усиление местных центров
  \item Развитие феодальных отношений
  \item Прекращение действия пути "из варяг в греки"
  \item Отсутствие внешней опасности
\end{enumerate}

Однако существоло три крупных политических центра:
\begin{enumerate}
  \item Новгородская республика
  \item Далицко-волынское княжество
  \item Ростово-Суздальское (поздне -- Владимирско-Суздальское) княжество
\end{enumerate}

\subsubection{Новгородское княжество}

Официально - демократическая республика, но на самом деле - олигархическая (300 золотых поясов). Для защиты республики приглашали князя извне, рода Рюриковичей, за оплату.

\subsubsection{Далицко-Волынское княжество}

Самое обжитое княжество. Князь был, но его власть была ограничена. Из-за того, что бояре владели большими землями (наравне с князем), а также множество боярев были связаны с родом Рюриковичей, любой боярин мог стать князем (князь был просто \textit{первым среди равных}).
Княжество начало расширяется, было захвачено Киевское княжество\ldots но потом пришлось сдаться Золотой Орде. 

\subsubsection{Ростово-Владимирское княжество}

В дальнейшем именно из этого княжества и произошла современная Россия.

\subsection{Князья Ростово-Владимирского княжества}

\subsubsection{Юрий Владимирович Долгорукий}

Ему не дали стать киевским князем, хотя в соответствии с лестничной системой наследования была его очередь. Его назначили Ростово-Владимирсим князем, но он "тянул"\ руки к Киеву, за что его и назвали "долгоруким".
Дважды захватил Киев -- первый раз неудачно, второй раз -- захватил, но сам был отравлен.

\subsubsection{Андрей Боголюбский}

Сын Владимира Долгорукого

\subsubsection{Всеволод Большое Гнездо}

Сводный брат Андрея Боголюбского. 12 детей, поэтому и получил произвище "большое гнездо".
Разделил княжество на 12 уделов, раздал детям.

\subsubsection{Юрий II Всеволодович}

При нём произошло первое монгольское нашествие, бежал и был убит.

\subsubsection{Ярослав Всеволодович}

\subsubsection{Александр Ярославович Невский}

Дважды спас Новгород от нашествия тевтонцев. Легенда, но автор не успел ничего записать :(.

\subsection*{Крестные походы}

В Европе была сильная нехватка земель. Безземельные сыновья феодалов (рыцари) начинали заниматься разбоем. Для решения этой проблемы Папа Римский придумал "крестные походы". В это время в Палестине была власть арабов (мусульмане). Под благими намерениями "спасти от неверной религии"\, грабили земли. Потом арабы дали отпор, стала нужна новая цель. Тогда православные христиане тоже стали "неверными"\, что и позволило тевтонцам напасть на Новгородское княжество.

\subsection*{Монголы}

Монголы были кочевниками. Активно занимались скотоводчеством, поэтому им требовались большие территории под \textit{пастбища}. Сначала воевали между собой, потом начинали захватывать другие земли. Но в какой-то момент стало понятно, что будет выгоднее захватывать земли и давать местным правителям вести хозяйство и брать с них дань золотом и серебром. Это оказалось очень хорошей идеей, и монголы сменили тактику завоеваний.

