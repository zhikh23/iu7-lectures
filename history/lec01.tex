\lesson{1}{История как наука}

\begin{definition}
  Исторический источник - результат или продукт целенаправленной человеческой деятельности, созданный в конкретную историческую эпоху.
\end{definition}

Виды исторических источников:
\begin{enumerate}
  \item Письменные: летописи, гос.акты, письма;
  \item Вещественные: орудия, амфоры;
  \item Аудиовизуальные.
\end{enumerate}

\subsection*{Методология}
\begin{itemize}
  \item Сравнительный метод: сопоставление ифнормации в разных источниках;
  \item Принципы историзма: оцениваем события применительно к эпохе (по достоинству);
  \item Принципы системности: описываем события применительно к эпохе (рассматриваем все обстоятельства эпохи);
  \item Наративный (описательный) метод;
  \item Метод периодизации.
\end{itemize}

\subsection*{Периодизация истории}

\begin{enumerate}
  \item \textbf{Антропогинез / Праистория} \\
    2 - 1.6 млн назад $\to$ 40 тыс. лет назад \\
    \textit{Процесс становления человека}

  \item \textbf{Первобытность} \\
    40 тыс. лет назад \\
    \textit{Возникновение первых государств}

  \item \textbf{Древний мир} \\
    5 тыс. лет назад - 476 год н.э. \\
    \textit{Падение Западно-римской империи}

  \item \textbf{Средние века} \\
    476 год н.э. - XVI вв

  \item \textbf{Новое время} \\
    XV - XVI - 1918 г. \\
    \textit{Перестроение мира}

  \item \textbf{Новейшее время} \\
    1918 г - наши дни
\end{enumerate}

\subsection*{Праистория}

Масштабный природный катаклизм заставил первобытных людей перемещаться. Начинается освоение орудий, а следовательно, и сознания.

Строгая иерархия:
\begin{enumerate}
  \item Групповой отбор: менее и более приспособленные начинают сотрудничать;
  \item Возникновение морали, табу - начало социогенеза, управление инстинктами;
  \item Появилось искусство, религия.
\end{enumerate}

\section{Возникновение древнерусского государства}
Существует несколько теория возникновений Древней Руси:
\begin{enumerate}
  \item \textbf{Норманская теория}: Древняя Русь возникла благодаря норманам.
  \item \textbf{Антинорманская теория}: существовали предпосылки для возникновения государства.
\end{enumerate}

