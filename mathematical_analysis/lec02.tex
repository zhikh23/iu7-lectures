\lesson{2}{Математический анализ}

\section{Числовая последовательность}

\begin{definition}
  \textbf{Числовая последовательность} - это \underline{бесконечное} множество числовых значений, которое можно упорядочить (перенумеровать).
\end{definition}

Задать последовательность - указать формулу или правило, по которой $\forall n\in \N$ можно записать соответствующий элемент последовательности.

\begin{note}
  Множество значений последовательности может быть конечным или бесконечным, но число число элементов последовательности всегда бесконечно.
\end{note}

\begin{eg}
  $1, -1, 1, -1, 1\ldots$
  \begin{itemize}
    \item Число элементов бесконечно
    \item Значенией последовательности два
  \end{itemize}
  \[
    x_n = (-1)^{n+1}
  .\] 

  $2, 2, 2, 2, 2\ldots$
  \begin{itemize}
    \item Число элементов бесконечно
    \item Значенией последовательности одно
  \end{itemize}
  \[
    x_{n} = 2*1^{n}
  .\] 
  $1, 2, 3, 4, 5\ldots$
  \begin{itemize}
    \item Число элементов бесконечно
    \item Значений последовательности бесконечно
  \end{itemize}
  \[
  x_{n} = n, \forall n\in \N
  .\] 
\end{eg}

\begin{definition}
  Последовательность чисел $\{x_{n}\}$ называется неубывающей, если каждый последующий член $x_{n+1} \ge x_{n}, \forall n\in \N$.
\end{definition}
\begin{eg}
  $1, 2, 3, 4, 4, 5, 5\ldots$
\end{eg}

\begin{definition}
  Последовательность чисел $\{x_{n}\} $ называется возрастающей, если каждый последующий член $x_{n+1} > x_{n}, \forall n \in \N$.
\end{definition}
\begin{eg}
  $1, 2, 3, 4, 5, 6, 7\ldots$
\end{eg}

\begin{definition}
  Последовательность чисел $\{x_{n}\} $ называется невозрастающей, если каждый последующий член $x_{n+1} \le x_{n}, \forall n \in \N$.
\end{definition}
\begin{eg}
  $\frac{1}{2}, \frac{1}{3}, \frac{1}{3}, \frac{1}{4}\ldots $
\end{eg}

\begin{definition}
  Последовательность чисел $\{x_{n}\} $ называется убывающей, если каждый последующий член $x_{n+1} < x_{n}, \forall n \in \N$.
\end{definition}
\begin{eg}
  $\frac{1}{2}, \frac{1}{3}, \frac{1}{4}, \frac{1}{5}\ldots$
\end{eg}

\begin{definition}
  Возрастающая, убывающая последовательности называются строго монотонными.
\end{definition}

\begin{definition}
  Неубывающая, возрастающая, невозрастающая, убывающая последовательности называются монотонными.
\end{definition}

Немонотонная последовательность:
\[
1, 2, 3, 2, 1\ldots
.\] 

Постоянная последовательность
\[
1, 1, 1, 1, 1\ldots
.\] 

\section{Предел последовательности}

\begin{definition}
Число $a$ называется пределом последовательности $\{x_{n}\} $, если для любого положительного числа $\epsilon$ найдется натуральное число  $N\left(\epsilon  \right) $, зависящее от $\epsilon$ такое что если порядковый номер $n$ члена последовательности станет больше $N(\epsilon)$, то имеет место неравенство  $|x_{n} - a| < \epsilon$.
\[
\lim_{x \to \infty} x_{n} = a \quad \iff \quad
(\forall \epsilon > 0)(\exists N(\epsilon) \in \N) : (\forall n > N(\epsilon)) \implies |x_{n}-a| < \epsilon
.\] 
\end{definition}
\begin{note}
  Т.е. начиная с номера $N(\epsilon) + 1$ все элементы последовательности $\{x_{n}\} $ попадают в $\epsilon$-окрестность точки $a$.
\end{note}

\subsection{Геометрический смысл}
\begin{align*}
|x_{n} - a| < \epsilon \\
- \epsilon< x_{n} - a < \epsilon \\
a - \epsilon < x_{n} < a + \epsilon \\
\forall n > N(\epsilon) \\
\end{align*}
% (тут числовая прямая)

Какой бы малый $\epsilon$ мы не взяли, все элементы последовательности $\{x_{n}\}$ попадают в $\epsilon$-окрестность точки $a$, причем чем меньше $\epsilon \downarrow$, тем  $N(\epsilon) \uparrow$.

\begin{eg}
  Рассмотрим последовательность $x_{n} = \frac{1}{n+1} = \{ \frac{1}{2}, \frac{1}{3}, \frac{1}{4}, \frac{1}{5}, \frac{1}{6}\ldots \} = \{x_1, x_2, x_3, x_4, x_5, x_6\ldots\} $
% Тут числовая прямая с элементами последовательности
  \[
  \lim_{n \to \infty} x_{n} = a
  .\] 
  \[
  \lim_{n \to \infty} \frac{1}{n+1} = 0
  .\] 
\end{eg}
\begin{eg}
  Пусть $\epsilon = 0.3$, $x_{n}\in (a-\epsilon; a+\epsilon)$, т.е. $(-0.3; 0.3)$ \\
  Получается два элемента $x_1, x_2 \not\in (-0.3, 0.3) \implies N(\epsilon) = 2$ \\
  $N(\epsilon) + 1 = 3$ \\
  $x_3., x_4, x_5 \ldots \in (-0.3, 0.3)$ \\
\end{eg}

\begin{definition}
  Последовательность, имеющая предел, назыается сходящейся.
\end{definition}

\begin{definition}
  Последовательность $\{x_{n}\} $ называется ограниченной снизу (сверху), если $\exists m \in \R (M \in \R)$, что для всех $\forall n \in \N$ выполнено неравенство $x_{n} \ge m$ ($x_{n} \le M$)
  % два рисунка - ограничена сверху и органичена снизу
\end{definition}

\begin{definition}
  Последовательность $x_{n}$ называется ограниченной, если она ограничена и сверху, и снизу, т.е. $\forall n \in \N, m \le x_{n} \le M$ или $|x_{n}| \le M$.
\end{definition}

\begin{definition}
  Последовательность $\{x_{n}\} $ называется фундаментальной, если для любого $\epsilon > 0$ $\exists $ свой порядковый номер $N(\epsilon)$ такой, что при всех $n \ge N(\epsilon)$ и $m \ge  N(\epsilon)$ выполнено неравенство $|x_{n} - x_{m}| < \epsilon$.
  \begin{align*}
    \forall \epsilon > 0 \exists N(\epsilon) \forall n \ge N(\epsilon), \forall m \ge N(\epsilon) \implies |x_{n} - x_{m}| < \epsilon
  \end{align*}
\end{definition}

\begin{theorem}
  Теорема (критерий Коши существования предела последовательности) \\
  Для того, чтобы последовательность была сходящейся, необходимо и достаточно она была фундаментальной.
  \[
    \{x_{n}\} \text{ -сходится} \iff \{x_{n}\} \text{- фундаментальная}
  .\] 
\end{theorem}

\subsection{Свойства сходящихся последовательность}

\begin{theorem}
  О существовании единственности предела последовательности \\
  Любая сходящаяся последовательность имеет единственный предел.
\end{theorem}
\begin{proof}
  Пусть $\{x_{n}\} $ - сходящаяся последовательность. \\
  Метод от противного. Пусть последовательность $\{x_{n}\} $ имеет 2 различных предела.
  \begin{equation}
    \lim_{n \to \infty} = a
  \end{equation}
  \begin{equation}
    \lim_{n \to \infty} = b
  \end{equation}
  \[
    a \neq b
  \]
\end{proof}

