\section{Формула Тейлора. Многочлен Тейлора}

\begin{theorem}
  Пусть функция $y = f(x)$ дифференцируема $n$ раз в точке $x_0$ и определена в некоторой окрестности этой точки. Тогда $\forall x \in S(x_0)$ имеет место формула Тейлора: \[
    f(x) = f(x_0) + \frac{f'(x_0)}{1!}(x - x_0) + \frac{f''(x_0)}{2!}(x - x_0)^2 + \ldots + \frac{f^{(n)}}{n!}(x - x_0)^n + R_n(x)
    \]
  Или кратко: $f(x) = P_n(x) + R_n(x)$, где:  \[
    P_n(x) = f(x_0) + \frac{f'(x_0)}{1!}(x - x_0) + \frac{f''(x_0)}{2!}(x - x_0)^2 + \ldots + \frac{f^{(n)}}{n!}(x - x_0)^n + R_n(x)
  \] 
\end{theorem}

\begin{itemize}
  \item $P_n(x)$ называют \textit{многочленом} или \textit{полиномом Тейлора}.
  \item $R_n(x)$ называют \textit{остаточным членов формулы Тейлора}.
\end{itemize}

\begin{proof}
  Покажем, что многочлен $P_n(x)$ существует. Будем искать многочлен Тейлора в виде:  \[
    P_n(x) = a_0 + a_1 (x - x_0) + a_2 (x - x_0)^2 + a_3 (x - x_0)^3 + \ldots + a_n (x - x_0)^n \tag{2} 
  \]
  где $a_1, a_2, a_3 \ldots a_n$ -- некоторые константы.

  Пусть выполнены условия: \[
    P_n(x_0) = f(x_0) \quad P_n'(x) = f'(x) \quad \ldots \quad P_n^{(n)}(x) = f^{(n)}(x) \tag{3} 
  \] 

  $f'(x_0), f''(x_0), \ldots f^{(n)}(x)$ существуют т.к. $y = f(x)$ дифференцируема $n$ раз в точке $x_0$.

  Вычислим $P_n'(x), P_n''(x), \ldots P_n^{(n)}(x)$:
  \begin{align*}
    P_n'(x) &= a_1 \cdot 1 + a_2 \cdot 2(x - x_0) + a_3 \cdot 3(x - 0)^2 + \ldots + a_n \cdot n (x - x_0)^{(n-1)} \\
    P_n''(x) &= a_2 \cdot 1 + a_3 \cdot 3 \cdot 2(x - x_0) \\
             &+ a_4 \cdot 4 \cdot 3(x - 0)^2 + \ldots + a_n \cdot n \cdot (n - 1)(x - x_0)^{(n-2)} \\
              & \quad \ldots \\
    P_n^{(n)}(x) &= a_n n (n - 1)(n - 2) \ldots 1 = a_n \cdot n!
  \end{align*}

  \begin{align*}
    &P_n(x_0) = a_0 = f(x_0) \\
    &P_n'(x_0) = 1 \cdot a_1 = f'(x_0) \\
    &P_n''(x_0) = 1 \cdot 2 \cdot a_2 = 2 f''(x_0)\\
    & \qquad \ldots \\
    &P_n^{(n)}(x_0) = n! a_n = n! \cdot f^{(n)}(x_0)
  \end{align*}

  Выразим $a_0, a_1, a_2, \ldots a_3$: \[
     a_0  = f(x_0) 
     \quad a_1 = \frac{f'(x_0)}{1!}
     \quad a_2 = \frac{f''(x_0)}{2!}
      \quad \ldots
    \quad a_n = \frac{f^{(n)}(x_0)}{n!}
  \]
  Подставим значения $a_1, a_2, a_3, \ldots a_n$ в (2): \[
    P_n(x) = f(x_0) + \frac{f'(x_0)}{1!}(x - x_0) + \frac{f''(x_0)}{2!}(x - x_0)^2 + \ldots + \frac{f^{(n)}}{n!}(x - x_0)^n + R_n(x)
  \] 
\end{proof}

\begin{theorem}
  Пусть функция $y = f(x)$ дифференцируема $n$ раз в точке $x_0$, тогда $x \to x_0$: \[
    \boxed{R_n(x) = o((x - x_0)^n)}
  \] 
  -- \textit{форма Пеано}. 
\end{theorem}
\begin{proof}
  Формула Тейлора:
  \begin{align*}
    f(x) &= P_n(x) - R_n(x) \\
    R_n(x) &= f(x) - P_n(x)
  \end{align*}
  В силу условия (3):
  \begin{align*}
    R_n(x) &= f(x_0) - P_n(x_0) = f(x_0) - f(x_0) = 0 \\
    R_n'(x) &= f'(x_0) - P_n'(x_0) = f'(x_0) - f'(x_0) = 0 \\
            &\ldots \\
    R_n^{(n)}(x) &= f^{(n)}(x_0) - P_n^{(n)}(x_0) = f^{(n)}(x_0) - f^{(n)}(x_0) = 0 \\
  \end{align*}

  Вычислим: 
  \begin{align*}
    \lim_{x \to x_0} \frac{R_n(x)}{(x - x_0)^n} &= \left( \frac{0}{0} \right) = \\
      &= \lim_{x \to x_0} \frac{R_n'(x)}{n(x - x_0)^{n-1}} \\
      &\ldots \\
      &= \lim_{x \to x_0} \frac{R^{(n)}}{n(n-1)(n-2) \ldots 1} \\
      &= \frac{1}{n!} \lim_{x \to x_0} R_n^{(n)}(x) = \frac{1}{n!} \cdot 0 = 0 
  \end{align*}
  Вывод: $Rn(x) = o((x - x_0)^n)$ при $x \to x_0$.
\end{proof}

